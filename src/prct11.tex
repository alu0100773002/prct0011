\documentclass{beamer}
\usepackage[spanish]{babel}
\usepackage[utf8]{inputenc}

\usetheme{Madrid}
\definecolor{MiVioleta}{RGB}{122,59,122}
\definecolor{MiAzul}{RGB}{0,88,147}
\definecolor{MiGris}{RGB}{56,61,66}
\setbeamercolor*{palette primary}{use=structure, fg=white, bg=MiVioleta}
\setbeamercolor*{palette secundary}{use=structure, fg=white, bg=MiAzul}
\setbeamercolor*{palette tertiary}{use=structure, fg=white, bg=MiGris}

\title[Num $\pi$]{Número $\pi$}
\author[Zuleica Reina Segura]{Zuleica Reina Segura}
\institute{Fac. Mat.}
\date[24-04-2014]{24 de abril de 2014}

\begin{document}

\begin{frame}

  \includegraphics[width=0.15\textwidth]{img/ullesc.eps}
  \hspace*{7.5cm}
  \includegraphics[width=0.16\textwidth]{img/fmatesc.eps}
  \titlepage

  \begin{scriptsize}
    \begin{center}
     Facultad de Matemáticas \\
     Universidad de La Laguna
    \end{center}
  \end{scriptsize}

\end{frame}


\begin{frame}
  \frametitle{Índice}  
  \tableofcontents[pausesections]
\end{frame}
 
\section{Introducción a la historia}


\begin{frame}
\frametitle{Historia del cálculo de $\pi$}
Nos encontramos con el número $\pi$ cuando dividimos
la longitud de una circunferencia entre su diámetro.
Los antiguos egipcios (hacia 1600 a. de C.) ya sabían 
que existía una relación entre la longitud de la circunferencia 
y su diámetro y entre el área del círculo y el diámetro al cuadrado.
En Mesopotamia, más o menos por la misma época, los babilonios utilizaban
el valor 3'125 según puede leerse en la Tablilla de Susa. 
\end{frame}

\begin{frame}
\frametitle{Historia del cálculo de $\pi$}
En China también se hicieron esfuerzos para calcular su valor.
Liu Hui en el siglo III, utiliza polígonos de hasta 3072 lados
para conseguir el valor de 3'14159, y Tsu Ch'ung Chi en el siglo V 
da como valor aproximado 3'1415929.
En 1429, Al-Khasi utiliza el método de Arquímedes y trabaja 
con polígonos de hasta 805.306.368 lados.
\end{frame}

\section{Cálculos con $\pi$}
  
\begin{frame}
\frametitle{Definición}
$\pi$ es la razón entre la longitud de cualquier circunferencia y la de su diámetro.
\begin{block}{Fórmulas que contienen el número $\pi$}
  \begin{itemize}
  \item
   Área del círculo: $\pi r^2$
  \pause

  \item
   Área del cilindro: $2\pi r(r+h)$
  \pause

  \item
   Área del cono: $\pi r^2 +\pi rg$
  \pause

  \item
   Área de la esfera: $4\pi r^2$
  \pause

  \item
   Área interior de la elipse con semiejes: $\pi ab$
  \pause

  \end{itemize}
\end{block}

\end{frame}
\section{Bibliografía}
\begin{frame}
  \frametitle{Bibliografía}

  \begin{thebibliography}{10}
    \beamertemplatebookbibitems
    \bibitem[Fórmula que contienen el número $/pi$]{Internet}  
    Página web. 
    (2014) 
   {\small $http://es.wikipedia.org/wiki/N$}

    \beamertemplatebookbibitems
    \bibitem[El número $\pi$]{Internet}  
    Página web
    (2014) 
   {\small $http://mimosa.pntic.mec.es/jgomez53/matema/conocer/numpi.htm$}

   \end{thebibliography}
\end{frame}

\end{document}
